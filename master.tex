\documentclass{beamer}

% Theme choice
\usetheme{Madrid}

% Title page details
\title{Slurm 101}
\subtitle{How to use the HPC-infrastructure at the CIMH}
\author{Moritz Werr}
\institute{Research IT}
\date{\today}

\begin{document}

% Title page frame
\begin{frame}
    \titlepage
\end{frame}

% Outline frame
\begin{frame}{Outline}
    \tableofcontents
\end{frame}

% Introduction frame
\section{Parallel Computing}
\begin{frame}{Parallel Computing}

Computer Programs contain multiple steps.
Some steps can only be run \textbf{serially}, some can be run in \textbf{parallel}.
Serial parts are hard to speed up. Parallel parts can be speed up fairly easy.

\begin{alertblock}{Careful!}
	Trying to parallelize serial steps can lead to a decrease in performance, crashes and wrong results!
\end{alertblock}

Knowing which part of a program can be parallelized requires knowledge of the workload and your program.
 Parallelization can cause \textbf{Race Conditions} and \textbf{Deadlocks}.

\begin{block}{Don't waste resources}
Simply adding more computational resources won't increase calculation speed.
\end{block}



%    \begin{itemize}
%        \item Programs have parallel and serial parts
%        \item increase speed of programs by adding more computational power to parallel parts
%        \item Split up larger problems into smaller independently solvable sub-problems to computate them in parallel
%    \end{itemize}
\end{frame}

\begin{frame}{Parallel Computing}
	There are multiple ways to parallelize your program:	
	\begin{enumerate}
		\item Vectorization
		\item \textbf{Multi-Processing}
		\item \textbf{Multi-Threading}
		\item Multi-Streaming with GPUs
	\end{enumerate}
	\begin{block}{Right tool for the job}
			Choosing the right tool for the job is important.  In these slides I will focus mainly on Multi-Processing and Multi-Threading.
	\end{block}

\end{frame}
\begin{frame}{Parallel Computing}
	Parallel Computation requires to split up a \textbf{larger problem} into \textbf{smaller sub-problems}, which can be calculated independently. May this be in a shared memory space or seperate memory spaces.
	
	\begin{block}{Vectors, Matrices, Tensors}
		Vectors, Matrices and Tensors operations can be parallelized very easily as each element can calculated independently from the others.
	\end{block}
	Splitting up a large datasets into smaller independent datasets is key for parallel execution. Each dataset can be computed on different Nodes inside the cluster.
\end{frame}

% Main content frame
\section{Our HPC-Cluster}
\begin{frame}{Our HPC-Cluster}
	
	    \begin{itemize}
		\item 24 virtual nodes and many other hardware nodes (old hardware getting reused)
		\item Access via your RDS-Machines
		\item Local storage and Flstorage+Home shared
		\item Resources are CPU, RAM and GPU
	\end{itemize}
%    \begin{block}{Key Point 1}
%        Describe the first key point here.
%    \end{block}

%    \begin{block}{Key Point 2}
%        Describe the second key point here.
%    \end{block}
\end{frame}

% Conclusion frame
\section{Basic Tools}
\begin{frame}{Basic Tools}
    \begin{itemize}
        \item sinfo/squeue Status
        \item salloc/srun/sbatch Nutzung
        \item sacct Analyse
    \end{itemize}
\end{frame}

\section{Requesting Resources}
\begin{frame}{Requesting Resources}
	\begin{itemize}
		\item Default Values
		\item Absolute/relative
		\item CPU/RAM/GPU
	\end{itemize}
\end{frame}

\section{Using the cluster}
\begin{frame}{Interactive Mode}
	\begin{itemize}
		\item Interactive mode for testing
		\item srun to run commands on cluster
		\item salloc for longer session
	\end{itemize}
\end{frame}



\begin{frame}{Batch Processing}
	\begin{itemize}
		\item Kinit -> copy input -> process -> Kinit -> mv output
		\item Request ressources in script
		\item Mail
	\end{itemize}
\end{frame}

\section{Kerberos}
\begin{frame}{Kerberos}
	\begin{itemize}
		\item Kerberos on RDS (Login -> no manual kinit needed)
		\item Create Keytab
		\item Kinit in script
	\end{itemize}
\end{frame}

\section{Examples}
\begin{frame}{Examples}
	\begin{itemize}
		\item Normal
		\item Array
	\end{itemize}
\end{frame}
	% End of presentation
\begin{frame}
    \centering \Large
    Thank You!
\end{frame}

\end{document}
